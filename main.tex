\documentclass{article}
\usepackage[
    backend=biber,
    style=numeric,
    sorting=none
]{biblatex}

\addbibresource{refs.bib}

\usepackage{geometry}
\usepackage{setspace}
\usepackage{hyperref}
\usepackage[english]{babel}
\usepackage{amsmath}
\usepackage{tcolorbox}
\usepackage{float}
\usepackage{graphicx}
\usepackage{caption}
\usepackage{csquotes}
\usepackage{subcaption}
\usepackage{todonotes}
\geometry{
    textheight=9in,
    textwidth=5.5in,
    top=1in,
    headheight=12pt,
    headsep=25pt,
    footskip=30pt
}
\doublespacing
\begin{document}
\begin{center}
\thispagestyle{empty}
\textbf{\LARGE Slovak University of Technology in Bratislava}
\par\end{center}{\Large \par}

\begin{center}
\textbf{\Large Faculty of Informatics and Information Technologies}
\par\end{center}{\Large \par}

\vfill{}

\begin{center}

\begin{center}
\textbf{\LARGE Large Language Models as Critics for data Quality}
\par\end{center}{\huge \par}

\medskip{}

\textbf{\Large Bc. Andrej Vyšný}
\par\end{center}{\Large \par}

\medskip{}

\vfill{}

\textbf{Study program:} Intelligent Software Systems

\textbf{Instructor:} Ing. William Brach

\textbf{Academic year:} 2025 / 2026
\thispagestyle{empty}
\newpage

\thispagestyle{empty}
\newpage

\section{Assignment}
\todo[inline]{Prepisat to EN}
\textbf{
Aplikácia metód umelej inteligencie do problematiky Web scrapingu}

S popularizáciou veľkých jazykových modelov (LLMs) sa otvorila otázka, ako udržať tieto modely aktuálne a zabezpečiť, aby ich znalosti zodpovedali súčasnému dianiu vo svete. Väčšina aktuálnych informácií sa totiž nachádza v online webovom prostredí. S rastúcou potrebou poskytovať aktuálne znalosti LLMs rastie aj potreba vytvárania, používania a udržiavania spoľahlivých web scraperov, ktoré zhromažďujú najnovšie informácie. Využitie umelej inteligencie (AI) v oblasti web scrapingu predstavuje nový prístup k získavaniu dát z internetu. Na rozdiel od tradičných metód web scrapingu poskytujú AI nástroje schopnosť dynamicky sa adaptovať a následne spracovávať neštruktúrované údaje. AI modely využívané v tejto oblasti sú založené na strojovom učení, konkrétne na spracovaní prirodzeného jazyka (NLP), ktoré položilo základy pre pokročilé techniky web scrapingu v rôznych odvetviach vrátane e-commerce, výskumu trhu a spravodajstva. Frameworky ako PyTorch, Huggingface a Langchain, DSPy umožňujú vývoj sofistikovaných AI modelov a agentov pre web scraping, zatiaľ čo nástroje ako Playwright a Beautiful Soup poskytujú robustné základy pre extrakciu dát. Nové nástroje na predspracovanie webových stránok, ako Jina Reader, FireCrawl alebo Crawl4AI, rozširujú možnosti web scrapingu pre využitie v spolupráci s LLM.
Analyzujte tieto technológie a porovnajte ich ekosystémy z pohľadu vývojára aj koncového používateľa, taktiež porovnajte tieto nástroje z hľadiska ich schopností a presnosti v extrakcii dát. Navrhnite a implementujte AI-powered web scraping nástroj využívajúci najnovšie technológie v oblasti strojového učenia. Rozoberte, ako sa tento prístup líši od tradičných metód web scrapingu. Nakoniec vyhodnoťte vami implemnetovaný nástroj z hľadiska presnosti, škálovateľnosti a schopnosti spracovávať dynamický webový obsah z rôznych oblastí.



\newpage

\section{Related work}

\section{Research target}

\section{Methodology}

\section{Evaluation}

\printbibliography
\end{document}
